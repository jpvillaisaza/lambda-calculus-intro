\documentclass{beamer}

\usetheme{default}

\setbeamertemplate{navigation symbols}{}

\usepackage{polyglossia}

\setdefaultlanguage[variant=usmax]{english}

\usepackage{tikz}
\usetikzlibrary{positioning}

\usepackage{fancyvrb}
\usepackage{relsize}

\DefineShortVerb{\|}
\DefineVerbatimEnvironment{code}{Verbatim}{fontsize=\relsize{-1},frame=lines}

\DeclareMathOperator{\fv}{FV}

\newcommand{\contract}{\ensuremath{\triangleright_{1\beta}}}
\newcommand{\contracts}[2]{\ensuremath{{#1}\,\contract\,{#2}}}
\newcommand{\reduce}{\ensuremath{\triangleright_{\beta}}}
\newcommand{\reduces}[2]{\ensuremath{{#1}\,\reduce\,{#2}}}

\title{Introduction to $\lambda$-calculus}
\date{April 13, 2016}

\begin{document}

%%%%%%%%%%%%%%%%%%%%%%%%%%%%%%%%%%%%%%%%%%%%%%%%%%%%%%%%%%%%%%%%%%%%%%

\frame{\titlepage}

%%%%%%%%%%%%%%%%%%%%%%%%%%%%%%%%%%%%%%%%%%%%%%%%%%%%%%%%%%%%%%%%%%%%%%

\begin{frame}

  \begin{center}
    \begin{tikzpicture}[every node/.style={inner sep=0pt,minimum size=1mm}]
      \node [circle,draw,fill] (1920) at (3.00,7.00) [label=left:1920] {};
      \node [circle,draw,fill] (1930) at (3.00,6.00) [label=left:1930] {};
      \node [circle,draw,fill] (1940) at (3.00,5.00) [label=left:1940] {};
      \node [circle,draw,fill] (1950) at (3.00,4.00) [label=left:1950] {};
      \node [circle,draw,fill] (1960) at (3.00,3.00) [label=left:1960] {};
      \node [circle,draw,fill] (1970) at (3.00,2.00) [label=left:1970] {};
      \node [circle,draw,fill] (1980) at (3.00,1.00) [label=left:1980] {};
      \node [circle,draw,fill] (1990) at (3.00,0.00) [label=left:1990] {};

      \node at (0.00,6.50) {Combinatory logic};
      \node at (0.00,5.75) {$\lambda$-calculus};
      \node at (0.00,5.25) {(Turing machines)};
      \node at (0.00,3.00) {Lisp};
      \node at (0.00,0.00) {Haskell};

      \node at (6.00,7.00) {Moses Schönfinkel};
      \node at (6.00,6.50) {Haskell Curry};
      \node at (6.00,5.75) {Alonzo Church};
      \node at (6.00,5.25) {(Alan Turing)};
      \node at (6.00,3.00) {John McCarthy};
      \node at (6.00,2.50) {Dana Scott};
      \node at (6.00,2.00) {Henk Barendregt};

      \draw [-] (1920) to (1990);
  \end{tikzpicture}
  \end{center}
\end{frame}

%%%%%%%%%%%%%%%%%%%%%%%%%%%%%%%%%%%%%%%%%%%%%%%%%%%%%%%%%%%%%%%%%%%%%%

\begin{frame}

  \begin{quote}
    $\lambda$-calculus and combinatory logic are rather like the
    chassis of a bus, which gives the bus essential support but is
    definitely not the whole bus. Just as the chassis gains its
    purpose from the purpose to which the whole bus is put, so
    $\lambda$-calculus and combinatory logic gain their main purpose
    as parts of other systems of higher-order logic and programming.
  \end{quote}
  \hfill ---Hindley and Seldin
\end{frame}

%%%%%%%%%%%%%%%%%%%%%%%%%%%%%%%%%%%%%%%%%%%%%%%%%%%%%%%%%%%%%%%%%%%%%%

\begin{frame}

  \begin{definition}[$\lambda$-term]
    \begin{itemize}
    \item
      Atom
      \begin{itemize}
      \item
        Variable: $x$, $y$, $z$, $\dots$
      \item
        Atomic constant
      \end{itemize}
    \item
      Application: $MN$
    \item
      Abstraction: $\lambda{x}.M$
    \end{itemize}
  \end{definition}
\end{frame}

%%%%%%%%%%%%%%%%%%%%%%%%%%%%%%%%%%%%%%%%%%%%%%%%%%%%%%%%%%%%%%%%%%%%%%

\begin{frame}

  \begin{examples}[$\lambda$-terms]
    \begin{enumerate}
    \item
      $\lambda{x}.xy$
    \item
      $(\lambda{y}.y)(\lambda{x}.xy)$
    \item
      $x\,(\lambda{x}.(\lambda{x}.x))$
    \item
      $\lambda{x}.yz$
    \end{enumerate}
  \end{examples}
\end{frame}

%%%%%%%%%%%%%%%%%%%%%%%%%%%%%%%%%%%%%%%%%%%%%%%%%%%%%%%%%%%%%%%%%%%%%%

\begin{frame}

  \begin{block}{Notation ($\lambda$-term)}
    \begin{itemize}
    \item
      $\lambda$-terms: $M$, $N$, $P$, $\dots$
    \item
      Variables: $x$, $y$, $z$, $\dots$
    \end{itemize}
    \begin{itemize}
    \item
      $MNPQ \equiv (((MN)P)Q)$
    \item
      $\lambda{x_{1}x_{2}\dots x_{n}}.M \equiv
      \lambda{x_{1}}.\lambda{x_{2}}.\dots\lambda{x_{n}}.M$
    \end{itemize}
  \end{block}
\end{frame}

%%%%%%%%%%%%%%%%%%%%%%%%%%%%%%%%%%%%%%%%%%%%%%%%%%%%%%%%%%%%%%%%%%%%%%

\begin{frame}

  \begin{block}{Informal interpretation ($\lambda$-term)}
    \begin{itemize}
    \item
      Function application: $MN$ \hfill ($\equiv M(N)$)
    \item
      Function: $\lambda{x}.M$
      \begin{itemize}
      \item
        $(\lambda{x}.M)N \equiv [N/x]M$
      \end{itemize}
    \end{itemize}
  \end{block}
\end{frame}

%%%%%%%%%%%%%%%%%%%%%%%%%%%%%%%%%%%%%%%%%%%%%%%%%%%%%%%%%%%%%%%%%%%%%%

\begin{frame}

  \begin{examples}[Informal interpretation ($\lambda$-term)]
    \begin{itemize}
    \item
      $\lambda{x}.x(xy)$
      \begin{itemize}
      \item
        $(\lambda{x}.x(xy))N \equiv N(Ny)$
      \end{itemize}
    \item
      $\lambda{x}.y$
      \begin{itemize}
      \item
        $(\lambda{x}.y)N \equiv y$
      \end{itemize}
    \end{itemize}
  \end{examples}
\end{frame}

%%%%%%%%%%%%%%%%%%%%%%%%%%%%%%%%%%%%%%%%%%%%%%%%%%%%%%%%%%%%%%%%%%%%%%

\begin{frame}

  \begin{definition}[Free and bound variables]
    \begin{itemize}
    \item
      Bound: $((xv)(\lambda{yz}.\underline{y}v))w$
    \item
      Bound and binding: $((xv)(\lambda{\underline{yz}}.yv))w$
    \item
      Free:
      $((\underline{xv})(\lambda{yz}.y\underline{v}))\underline{w}$
    \end{itemize}
  \end{definition}
\end{frame}

%%%%%%%%%%%%%%%%%%%%%%%%%%%%%%%%%%%%%%%%%%%%%%%%%%%%%%%%%%%%%%%%%%%%%%

\begin{frame}

  \begin{example}[Free and bound variables]
    \begin{itemize}
    \item
      $P \equiv
      (\lambda{y}.y\underline{x}(\lambda{x}.y(\lambda{y}.\underline{z})x))\underline{vw}$
      \begin{itemize}
      \item
        $\fv(P) = \{x, z, v, w\}$
      \end{itemize}
    \end{itemize}
  \end{example}
\end{frame}

%%%%%%%%%%%%%%%%%%%%%%%%%%%%%%%%%%%%%%%%%%%%%%%%%%%%%%%%%%%%%%%%%%%%%%

\begin{frame}

  \begin{definition}[Closed term]
    \begin{itemize}
    \item
      A term without any free variables: $\fv(P) = \emptyset$
    \end{itemize}
  \end{definition}
\end{frame}

%%%%%%%%%%%%%%%%%%%%%%%%%%%%%%%%%%%%%%%%%%%%%%%%%%%%%%%%%%%%%%%%%%%%%%

\begin{frame}

  \begin{definition}[Substitution]
    \begin{itemize}
    \item
      $[N/x]M$ is the result of substituting $N$ for every free
      occurrence of $x$ in $M$, and changing bound variables to avoid
      clashes.
    \end{itemize}
  \end{definition}
\end{frame}

%%%%%%%%%%%%%%%%%%%%%%%%%%%%%%%%%%%%%%%%%%%%%%%%%%%%%%%%%%%%%%%%%%%%%%

\begin{frame}

  \begin{examples}[Substitution]
    \begin{itemize}
    \item
      $[w/x](\lambda{y}.x) \equiv \lambda{y}.w$
    \item
      $[w/x](\lambda{w}.x) \equiv \lambda{z}.w$ \hfill ($\not\equiv
      \lambda{w}.w$)
    \end{itemize}
  \end{examples}
\end{frame}

%%%%%%%%%%%%%%%%%%%%%%%%%%%%%%%%%%%%%%%%%%%%%%%%%%%%%%%%%%%%%%%%%%%%%%

\begin{frame}

  \begin{definition}[$\beta$-contracting, $\beta$-reducing]
    \begin{align*}
      \underbrace{(\lambda{x}.M)N}_{\beta\text{-redex}}
      &\underbrace{\contract}_{\beta\text{-contracts}}
      \underbrace{[N/x]M}_{\phantom{\beta}\text{contractum}}\\ {}\\ P\,&\underbrace{\contract
        \dots \contract}_{\underbrace{\reduce}_{\beta\text{-reduces}}}
      \, Q
    \end{align*}
  \end{definition}
\end{frame}

%%%%%%%%%%%%%%%%%%%%%%%%%%%%%%%%%%%%%%%%%%%%%%%%%%%%%%%%%%%%%%%%%%%%%%

\begin{frame}

  \begin{examples}[$\beta$-contracting, $\beta$-reducing]
    \begin{itemize}
    \item
      $\contracts{(\lambda{x}.x(xy))N}{N(Ny)}$
    \item
      $\contracts{(\lambda{x}.y)N}{y}$
    \end{itemize}
  \end{examples}
\end{frame}

%%%%%%%%%%%%%%%%%%%%%%%%%%%%%%%%%%%%%%%%%%%%%%%%%%%%%%%%%%%%%%%%%%%%%%

\begin{frame}

  \begin{example}[$\beta$-contracting, $\beta$-reducing]
    \begin{align*}
      (\lambda{x}.(\lambda{y}.yx)z)v
      &\quad\contract\phantom{\equiv} [v/x]((\lambda{y}.yx)z)\\
      &\quad\equiv\phantom{\contract} (\lambda{y}.yv)z\\
      &\quad\contract\phantom{\equiv} [z/y](yv)\\
      &\quad\equiv\phantom{\contract} zv
    \end{align*}
  \end{example}
\end{frame}

%%%%%%%%%%%%%%%%%%%%%%%%%%%%%%%%%%%%%%%%%%%%%%%%%%%%%%%%%%%%%%%%%%%%%%

\begin{frame}

  \begin{example}[$\beta$-contracting, $\beta$-reducing]
    \begin{align*}
      \Omega \equiv (\lambda{x}.xx)(\lambda{x}.xx)
      &\quad\contract\quad (\lambda{x}.xx)(\lambda{x}.xx)\\
      &\quad\contract\quad (\lambda{x}.xx)(\lambda{x}.xx)\\
      &\quad\dots
    \end{align*}
  \end{example}
\end{frame}

%%%%%%%%%%%%%%%%%%%%%%%%%%%%%%%%%%%%%%%%%%%%%%%%%%%%%%%%%%%%%%%%%%%%%%

\begin{frame}

  \begin{example}[$\beta$-contracting, $\beta$-reducing]
    \begin{align*}
      (\lambda{x}.xxy)(\lambda{x}.xxy)
      &\quad\contract\quad (\lambda{x}.xxy)(\lambda{x}.xxy)y\\
      &\quad\contract\quad (\lambda{x}.xxy)(\lambda{x}.xxy)yy\\
      &\quad\dots
    \end{align*}
  \end{example}
\end{frame}

%%%%%%%%%%%%%%%%%%%%%%%%%%%%%%%%%%%%%%%%%%%%%%%%%%%%%%%%%%%%%%%%%%%%%%

\begin{frame}

  \begin{definition}[$\beta$-normal form]
    \begin{itemize}
    \item
      A term $Q$ which contains no $\beta$-redexes.
    \end{itemize}
  \end{definition}
\end{frame}

%%%%%%%%%%%%%%%%%%%%%%%%%%%%%%%%%%%%%%%%%%%%%%%%%%%%%%%%%%%%%%%%%%%%%%

\begin{frame}

  \begin{example}[$\beta$-normal form]
    \begin{itemize}
    \item
      $zv$ is a $\beta$-normal form of $(\lambda{x}.(\lambda{y}.yx)z)v$.
    \end{itemize}
  \end{example}
\end{frame}

%%%%%%%%%%%%%%%%%%%%%%%%%%%%%%%%%%%%%%%%%%%%%%%%%%%%%%%%%%%%%%%%%%%%%%

\begin{frame}

  \begin{examples}
    \begin{itemize}
    \item
      $L \equiv (\lambda{x}.xxy)(\lambda{x}.xxy)$
      \begin{itemize}
      \item
        $L \,\contract\, Ly \,\contract\, Lyy \,\contract \,\dots$
      \end{itemize}
    \item
      Let $P \equiv (\lambda{u}.v)L$:
      \begin{itemize}
      \item
        $P \,\contract\, [L/u]v \equiv v$
      \item
        $P \,\contract\, (\lambda{u}.v)(Ly) \,\contract\,
        (\lambda{u}.v)(Lyy) \,\contract\, \dots$
      \end{itemize}
    \end{itemize}
  \end{examples}
\end{frame}

%%%%%%%%%%%%%%%%%%%%%%%%%%%%%%%%%%%%%%%%%%%%%%%%%%%%%%%%%%%%%%%%%%%%%%

\begin{frame}

  \begin{examples}[$\beta$-normal form]
    \begin{itemize}
    \item
      $\Omega \equiv (\lambda{x}.xx)(\lambda{x}.xx)$
    \item
      $T \equiv (\lambda{u}.v)\Omega$
    \end{itemize}
  \end{examples}
\end{frame}

%%%%%%%%%%%%%%%%%%%%%%%%%%%%%%%%%%%%%%%%%%%%%%%%%%%%%%%%%%%%%%%%%%%%%%

\begin{frame}

  \begin{example}[$\beta$-normal form]
    \begin{align*}
      (\lambda{x}.(\lambda{y}.yx)z)v
      &\quad\contract\quad (\lambda{y}.yv)z\\
      &\quad\contract\quad zv\\
      (\lambda{x}.(\lambda{y}.yx)z)v
      &\quad\contract\quad (\lambda{z}.zx)v\\
      &\quad\contract\quad zv
    \end{align*}
  \end{example}
\end{frame}

%%%%%%%%%%%%%%%%%%%%%%%%%%%%%%%%%%%%%%%%%%%%%%%%%%%%%%%%%%%%%%%%%%%%%%

\begin{frame}

  \begin{theorem}[Church-Rosser theorem for \reduce]
    \begin{center}
      \begin{tikzpicture}[
          node distance=3cm,
          every node/.style={
            draw,
            circle,
            inner sep=0pt,
            minimum size=1.5mm
          }
        ]

        \node [fill] (P)                    [label=$P$]                {};
        \node [fill] (M) [below left of=P]  [label=left:$M$]           {};
        \node [fill] (N) [below right of=P] [label=right:$N$]          {};
        \node        (T) [below right of=M] [label=below:$\exists{T}$] {};

        \draw [->]        (P) to (M);
        \draw [->]        (P) to (N);
        \draw [->,dashed] (M) to (T);
        \draw [->,dashed] (N) to (T);
      \end{tikzpicture}
    \end{center}
  \end{theorem}
\end{frame}

%%%%%%%%%%%%%%%%%%%%%%%%%%%%%%%%%%%%%%%%%%%%%%%%%%%%%%%%%%%%%%%%%%%%%%

\begin{frame}[fragile]

  \begin{example}[\texttt{Core}]
  \begin{code}
type CoreExpr = Expr Var

data Expr b
  = Var   Id
  | Lit   Literal
  | App   (Expr b) (Arg b)
  | Lam   b (Expr b)
  | Let   (Bind b) (Expr b)
  | Case  (Expr b) b Type [Alt b]
  | Cast  (Expr b) Coercion
  | Tick  (Tickish Id) (Expr b)
  | Type  Type
  | Coercion Coercion
  \end{code}
  \end{example}
\end{frame}

%%%%%%%%%%%%%%%%%%%%%%%%%%%%%%%%%%%%%%%%%%%%%%%%%%%%%%%%%%%%%%%%%%%%%%

\begin{frame}
  \frametitle{Bibliography}

  \begin{thebibliography}{}
  \setbeamertemplate{bibliography item}[book]
  \bibitem{}
    Hindley and Seldin (2008).
    \newblock \emph{Lambda-Calculus and Combinators: An Introduction}.
    \newblock Cambridge.
  \end{thebibliography}
\end{frame}

%%%%%%%%%%%%%%%%%%%%%%%%%%%%%%%%%%%%%%%%%%%%%%%%%%%%%%%%%%%%%%%%%%%%%%

\end{document}
